\documentclass{article} 
\usepackage{polski} %moze wymagac dokonfigurowania latexa, ale jest lepszy niż standardowy babel'owy [polish] 
\usepackage[utf8]{inputenc} 
\usepackage[OT4]{fontenc} 
\usepackage{graphicx,color} %include pdf's (and png's for raster graphics... avoid raster graphics!) 
\usepackage{url} 
\usepackage[pdftex,hyperfootnotes=false,pdfborder={0 0 0}]{hyperref} %za wszystkimi pakietami; pdfborder nie wszedzie tak samo zaimplementowane bo specyfikacja nieprecyzyjna; pod miktex'em po prostu nie widac wtedy ramek


% Zmiana rozmiarów strony tekstu
\addtolength{\voffset}{-1cm}
\addtolength{\hoffset}{-1cm}
\addtolength{\textwidth}{2cm}
\addtolength{\textheight}{2cm}

%bardziej zyciowe parametry sterujace rozmieszczeniem rysunkow
\renewcommand{\topfraction}{.85}
\renewcommand{\bottomfraction}{.7}
\renewcommand{\textfraction}{.15}
\renewcommand{\floatpagefraction}{.66}
\renewcommand{\dbltopfraction}{.66}
\renewcommand{\dblfloatpagefraction}{.66}
\setcounter{topnumber}{9}
\setcounter{bottomnumber}{9}
\setcounter{totalnumber}{20}
\setcounter{dbltopnumber}{9}

% własny bullet list z malymi odstepami
\newenvironment{tightlist}{
\begin{itemize}
  \setlength{\itemsep}{1pt}
  \setlength{\parskip}{0pt}
  \setlength{\parsep}{0pt}}
{\end{itemize}}

%obrazkow szukamy w nastepujacym katalogu:
\graphicspath{{pics/}}



%\title{Sprawozdanie z laboratorium:\\Metaheurystyki i Obliczenia Inspirowane Biologicznie}
%\author{}
%\date{}


\begin{document}

\thispagestyle{empty} %bez numeru strony

\begin{center}
{\large{Sprawozdanie z laboratorium:\\
Bioinformatyka\\
(szablon)}}

\vspace{3ex}

Część I: Analiza teoretyczna
%Część II: Algorytmy optymalizacji lokalnej i globalnej, problem QAP
%Część III: Eksperyment: ... (prezentację można zrobić w LaTeX - służy do tego klasa "beamer")

\vspace{3ex}
{\footnotesize\today}

\end{center}


\vspace{10ex}

Prowadzący: prof. dr hab.~inż. Marta Kasprzak

\vspace{5ex}

Autorzy:
\begin{tabular}{lllr}
\textbf{Damian Jurga} & inf..... & I2 & jasiu@serwer.domena.poczta.pl \\
\textbf{Grzegorz Miebs} & inf122453 & I2 & grzegorz.miebs@student.put.poznan.pl \\
\end{tabular}

\vspace{5ex}

Zajęcia środowe, 11:45.

\vspace{35ex}

\noindent Oświadczamy, że niniejsze sprawozdanie zostało przygotowane wyłącznie przez powyższych autorów,
a wszystkie elementy pochodzące z innych źródeł zostały odpowiednio zaznaczone i~są cytowane w bibliografii.  

\newpage




\section{Wstęp}
Celem tego sprawozdania jest przedstawienie teoretycznego opracowania metody heurystycznej rozwiązującej problem sekwencjonowania łańcuchów DNA z błędami pozytywnymi oraz negatywnymi w czasie wielomianowym. Alogrytm mając dany na wejściu zbiór oligonukleotydów (tj. ciągów nukleotydów: adeniny, tyminy, guaniny i cytozyny), długość sekwencji oryginalnej, powinien zwrócić sekwencję jak najbardziej zbliżoną do oryginalnej.

\section{Sformułowanie problemu}
Elementy zbioru oligonukleotydów $S$ zostały zakodowane jako słowa o długości $l$ nad alfabetem nukleotydów $\{A, C, G, T\}$.
Moc zbioru $S$ wynosi $n$.

W przypadku idealnym kolejne słowa pokrywają się na $l-1$ pozycjach; innymi słowy słowa powstałe przez usunięcie odpowiednio pierwszej lub ostatniej litery z dwóch następujących po sobie elementów sekwencji wynikowej są równe.
W takim wypadku sekwencja wyjściowa ma długość równą mocy zbiory $S$.

Ze względu na charakter źródła zbiorów wejściowych, $S$ może zawierać błędy dwojakiej natury.
Pojawienie się dodatkowych słów, nieobecnych w przypadku idealnym, nazywa się błędem pozytywnym, a brakujące dane (tj. brak słów bezpośrednio następnych, brak powtórzeń słów) --- negatywnym.
W przypadku obecności jedynie błędów negatywnych sekwencja wyjściowa nie jest dłuższa niż $n$ i zawiera wszystkie słowa z $S$; alternatywnie, w obecności jedynie błędów pozytywnych, sekwencja ma długość $n$ i zawiera $n-l+1$ słów z $S$.
W przypadku ogólnym, powinna zawierać maksymalną liczbę słów z $S$ oraz nie powinna być dłuższa niż $n$.

Problem można przedstawić jako skierowany graf pełny $G=(S,\lbrace(s_i,s_j):(s_i,s_j)\in S\times S \land s_i\neq s_j\rbrace)$, który z każdym wierzchołkiem ma utożsamiony  zysk równy $1$, a z każdym łukiem --- koszt odpowiadający liczbie niepokrywających się pozycji $k$ (w przypadku idealnym słowa pokrywają się na $l-1$ pozycjach, a ogólnie na $l-k$; $k\in\lbrace1,2,...,l\rbrace$).
Jako taki znalezienie najlepszej sekwencji sprowadza się do rozwiązania wariantu selektywnego problemu komiwojażera nie wymagające utworzenia cyklu, a jedynie ścieżkę.

\section{Algorytm}
\subsection{Opis}
Do rozwiązania tego problemu zbudujemy graf, którego wierzchołkami będą słowa ze zbioru $S$, a wartości łuku między wierzchołkami będą równe przesunięciu między tymi słowami. Przykładowo łuk z wierzchołka ACCGT do wierzchołka CCGTC będzie miał wartość 1 a do wierzchołka GTCGT wartość 3. Na skonstruowanym w ten sposób grafie rozwiązujemy problem komiwojażera maksymalizujący liczbę odwiedzonych wierzchołków przy ograniczeniu na sumę wartości wykorzystywanych łuków, która nie może być większa od $n - l$, gdyż w przeciwnym razie na wyjściu otrzymamy sekwencję dłuższą niż oryginalna.
Aby ograniczyć ponowne odwiedzanie tych samych wierzchołków, będziemy zwiększać wartość łuków prowadzących do odwiedzonych już wierzchołków o pewną stałą $C$.
Do rozwiązania problemu komiwojażera posłużymy się przeszukiwaniem wiązkowym oraz algorytmem wspinaczki.
\subsection{Lista kroków}
\begin{enumerate}
    \item Zbudowanie grafu
    \item Zaczynamy z losowego wierzchołka
    \item Znajdujemy $k$ najbliższych miast
    \item Do każdej z $k$ dotychczasowych ścieżek liczymy odległość po dodaniu każdego z wierzchołków z uwzględnieniem kary za powtórne odwiedzenie tego samego wierzchołka.Wybieramy $k$ najkrótszych ścieżek
    \item Powtarzamy krok 4 aż koszt ścieżek przekroczy krytyczną wartość $n-l$
    \item Dla każdej z $k$ wygenerowanych ścieżek sprawdzamy które przestawienie parami pozwoli maksymalnie zmniejszyć koszt ścieżki i wykonujemy je
    \item Powtarzamy krok 6 aż do uzyskania lokalnego optimum
    \item Ponownie wykonujemy przeszukiwanie wiązkowe
\end{enumerate}
\subsection{Złożoność obliczeniowa}
Złożoność pierwszego przeszukiwania wiązkowego jest równa $O(k*|S|^2)$ \\
Złożoność algorytmu wspinaczkowego (jeszcze nie wiem) \\
Złożoność drugiego przeszukiwania wiązkowego $O(k*|S|^2)$
\end{document}

